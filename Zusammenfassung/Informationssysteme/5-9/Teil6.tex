\subsubsection{Störungsmanagement und Problemmanagement}
Prozessmodell Störungsmanagement

\begin{figure}[h]
\centering
\includegraphics[width=0.8\textwidth]{assets/Störungsmanagement.png}
\end{figure}

Problemmanagement

\begin{itemize}
  \item Durchführung tiefgehender Diagnosen zur Problemlösung
  \item tiefes technisches Verständnis für IS-Architektur und IS nötig
  \item Problemdatenbank als zentraler Baustein
\end{itemize}

Vorgehensweisen beim Problemmanagement

\begin{itemize}
  \item unbekannte Probleme werden in Datenbank erfasst und analysiert
  \item Klassifikation des Problems
  \item Diagnose
  \item Fehlerbehandlung
  \item Dokumentation $\rightarrow$ unbekanntes Problem wird zu bekanntem Fehler
\end{itemize}

\subsubsection{Betrieb von Legacy Systemen}

\begin{itemize}
  \item veraltete Hardware und oder Technologien
  \item unterschiedlichste Gründe, dass bisher nicht ersetzt/modernisiert wurde
  \item \textbf{Ballast} und \textbf{Bremse} für Modernisierung und Optimierung
  \item zumeist wichtige (Kern-)Komponente für regelmäßige Aufgaben
  \item Ziel sollte sein, früh zu investieren
  \item End-of-life anstreben $\rightarrow$ Außerbetriebnahme mit Nachfolger oder Wegfall der Aufgabe
  \item Zusammenarbeit über definierte Schnittstelle (nicht unbedingt einfach)
\end{itemize}

\clearpage
\subsubsection{Containerisierung mit Docker}

\begin{itemize}
  \item Container als leichtgewichtige Alternative zu VM's
  \item Dockerfile (Image) und Compose-Datei (Services, Abhängigkeiten zw. Containern) beliebig komplex
  \item mit Docker-Compose kann eine Anwendung aus mehreren Containern bestehen, bspw. Services oder Datenbankserver die in eigenen Containern laufen
  \item bietet die Möglichkeit legacy apps auf moderner Hardware zu betreiben
  \item Abschottung von Containerschichten ermöglicht (temporären) Weiterbetrieb veralterter unsicherer Software
  \item Kubernetes als Verwaltungstool für Container $\rightarrow$ Rechenzentren müssen viele Container verwalten
\end{itemize}