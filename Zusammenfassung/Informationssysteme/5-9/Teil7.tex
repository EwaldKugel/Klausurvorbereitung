\subsubsection{ERP Grundlagen}
Enterprise Resource Planning

\begin{itemize}
  \item Personal und Ressourcen (Kapital, Betriebsmittel, Material, Informations- und Kommunikationstechnik) rechtzeitig bedarfsgerecht planen steuern und verwalten
  \item um effizienten betrieblichen Wertschöpfungsprozess und optimierte Steuerung betrieblicher Ablaufe zu gewährleisten
\end{itemize}

ERP-Systeme

\begin{itemize}
  \item unterstützen betriebliche Geschäftsvorfälle (Rechnungswesen Personalwesen, Materialwirtschaft, Produktion, Vertrieb)
  \item \textbf{Nutzenpotentiale} $\rightarrow$ Prozessstandardisierung, kostengünstiger Funktionsumfang, technische Innovationen
  \item \textbf{Risiken} $\rightarrow$ Umstellungsaufwand, Schulungsbedarf, Datenübernahme aus Altsystem
\end{itemize}

\clearpage
\subsubsection{SOA und Standardsoftware}
Serviceorientierte Architektur

\begin{itemize}
  \item verteilte Informationsarchitektur
  \item dynamischer Aufruf von anwendungsnahen, in sich abgeschlossenen Diensten
  \item Realisierung von lose gekoppelten, verteilten Anwendungssystemen
  \item bietet Integration mit anderen betrieblichen IS
  \item \textbf{Webservice} $\rightarrow$ Softwaredienst über offene Protokolle und Standardformate (XML), Vertriebsweg Internet
  \item \textbf{Cloud-Computing} $\rightarrow$ Anbieten von skalierbaren Webservices über mehrere verteilte Server, Cloud $\equiv$ Internet 
\end{itemize}

Standardsoftware

\begin{itemize}
  \item spart Zeit und Kosten
  \item Gesamtsystem wird in Teilschritten eingeführt (phasing)
  \item Anpassung an betriebsindividuelle Bedürfnisse durch Geschäftsprozessmodellierung, Customizing und Ergänzungsprogrammierung
  \item \textbf{mandantenfähig} $\rightarrow$ eine Installation, mehrere Kunden, getrennte Einstellungen, kein gegenseitiger Einblick in Daten
\end{itemize}